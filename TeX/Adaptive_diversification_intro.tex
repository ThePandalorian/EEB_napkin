\chapter{Adaptive Diversification}
Adaptive diversification refers to the phenomenon an initially monomorphic population evolving to become polymorphic due to (disruptive) frequency-dependent selection induced by ecological interactions~\citep{doebeli_adaptive_2011}. The framework has been widely used in evolutionary ecology and evolutionary game theory, especially in the context of sympatric speciation \citep{dieckmann_origin_1999}. Models of adaptive diversification can be broadly classified into three classes of models which differ starkly in their approach and assumptions \citep{doebeli_adaptive_2011}. I introduce all three approaches below.

\section{The `classic' adaptive dynamics approach}
The classic approach to modelling adaptive diversification was first articulated in the late 90s by J.J. Metz's group \citep{geritz_evolutionarily_1998}. The crux of the approach relies on the observation that evolutionary stability (in the ESS sense of evolutionary game theory) and asymptotic convergence stability (in the sense of dynamical systems) need not always coincide.\\
More concretely, consider an infinitely large asexual population that is monomorphic for some quantitative trait such that every individual of the population has the trait value $x \in \mathcal{T} \subseteq \mathbb{R}^n$. We are interested in the dynamics of $x$ over evolutionary time. There is assumed to be a separation of timescales between ecology and evolution such that whenever we observe the population, it is at ecological equilibrium (This is equivalent to a strong-selection + weak-mutation setting). The evolutionary dynamics of the trait value in the population are modelled as following the equation:
\begin{equation}
	\label{AD_canonical_eqn}
	\frac{dx}{dt} = g(x) = B(x)\left(\nabla_y f(y;x) \bigg{|}_{y=x}\right)
\end{equation}
Here, $B(x)$ describes the mutational process of the trait and is intended to model mutational biases. For the sake of simplicity, I will assume that $B(x) = 1$ below, but the essential results are not greatly changed by more complicated forms. $f(y;x)$ is the \emph{invasion fitness} function, and describes the expected fitness of a mutant type $y$ appearing in a population that is monomorphic $x$-valued.%\footnote{In evolutionary game theory notation, this is exactly the incentive function $h_{x,y,0}$ that characterizes whether a strategy is an ESS. Recall that $h_{p,q,\epsilon} = \mathbb{E}[p;(1-\epsilon)\delta_p+\epsilon\delta_q] - \mathbb{E}[q;(1-\epsilon)\delta_p+\epsilon\delta_q]$ is the `incentive' of switching from strategy $q$ to strategy $p$ in a population where a fraction $\epsilon$ of the individuals are $q$-players and the rest are $p$-players. Here, the strategies are continuous and the incentive is in terms of fitness.}
 $x$ is typically called the `resident' trait value. Thus, evolution is modelled as following the gradient $\nabla_y f(y;x)$ of the invasion fitness, a quantity sometimes called the selection gradient. \eqref{AD_canonical_eqn} can be interpreted to mean that at each time step, the population `samples' all local mutations, and if a mutant can invade this resident population, then the trait rapidly spreads in the population, and by the time we next `observe' it, the entire population has adopted this mutant trait. Equation \eqref{AD_canonical_eqn} is called the \emph{canonical equation of adaptive dynamics}. Fixed points for this equation, \textit{i.e} points $x^*$ for which $g(x^*)=0$, are called \emph{evolutionary singularities}. These singularities can be characterized by two different stability notions, as we will see below, and the difference between the two notions captures the important phenomenon of \textit{evolutionary branching}, which we will encounter shortly.

\definition{\emph{(Convergence stability)}}{ A singularity $x^*$ is said to be convergent stable (CS) if it is an asymptotically stable fixed point for equation \eqref{AD_canonical_eqn}.}

Thus, singularities which are convergence stable are local attractors, in the sense that nearby populations will evolve towards this state.

\definition{\emph{(Evolutionary stability)}}{ A singularity $x^*$ is said to be evolutionarily stable (ES) if it cannot be invaded by any nearby mutants.}
\\
\\
In the one dimensional case, evolutionary stability requires that the invasion fitness be such that no mutant can invade the population. Since $g(x^*) = \frac{\partial}{\partial y}f(y;x^*) = 0$ by definition at a singularity, the condition for evolutionary stability is controlled by the second derivative, and is given by:
\begin{equation}
	\label{ESS_condition}
	\frac{\partial^2}{\partial y^2} f(y;x) \bigg{|}_{y=x=x^*} < 0
\end{equation}
On the other hand, from elementary non-linear dynamics, we know that convergence stability requires $\frac{dg}{dx} < 0$, \textit{i.e}:
\begin{equation}
	\label{CS_condition}
	\frac{\partial^2}{\partial x \partial y} f(y;x) \bigg{|}_{y=x=x^*} + \frac{\partial^2}{\partial y^2} f(y;x) \bigg{|}_{y=x=x^*} < 0
\end{equation}
Note that neither \eqref{ESS_condition} nor \eqref{CS_condition} imply the other, and thus, we arrive at the following classification of evolutionary singularities:
\begin{center}
	\begin{tabularx}{0.4\textwidth}{ 
			| >{\centering\arraybackslash}X 
			| >{\centering\arraybackslash}X 
			| >{\centering\arraybackslash}X | }
		\hline
		& \textbf{ES} & \textbf{not ES} \\
		\hline
		\textbf{CS} &  \circled{A}  &  \circled{B} \\ 
		\hline
		\textbf{not CS} & \circled{C}  & \circled{D} \\
		\hline
	\end{tabularx}
\end{center}

Points which are not convergent stable are not of interest to us because populations can only attain such a state if they begin there. Thus, \circled{C} and \circled{D} can be ignored for our purposes.\footnote{Singularities of type \circled{C} are sometimes called `garden of Eden' points, since a population that begins at such a point will remain there and cannot be invaded by nearby mutants, but if a population does not begin there (or is somehow driven out by external forces), it can never find its way back to the singularity.}\\
Points of type \circled{A} are both evolutionarily stable and convergent stable. Such points represent `endpoints' for evolution, since populations are attracted to such points and also cannot evolve away from them since they cannot be invaded by any nearby mutants. Points of type \circled{B} are CS but not ES. Populations are attracted to such points, but once they have arrived, they are susceptible to invasion by nearby mutants. Let $x^*$ be such a point, and let $x_1 < x^* < x_2$ be two points in the immediate vicinity of $x^*$.
\begin{claim}{(Protected Polymorphism)}
	Each of $x_1$ and $x_2$ can invade the other
\end{claim}
\begin{proof}
	We can Taylor expand the invasion fitness function as
	\begin{align*}
		f(x_1 ; x_2) &= \underbrace{f(x_2 ; x_2)}_{=0} + \underbrace{(x_1 - x_2)}_{< 0 }\underbrace{\frac{\partial f}{\partial x_1}(x_1 ; x_2)\bigg{|}_{x_1 = x_2}}_{=g(x_2)} + \frac{(x_1-x_2)^2}{2} \frac{\partial^2 f}{\partial x_1^2}(x_1 ; x_2)\bigg{|}_{x_1 = x_2}
	\end{align*}
	where we have neglected terms that are higher than second order. Since $x^*$ is convergent stable, $\frac{d g}{dx}\bigg{|}_{x=x^*} < 0$ and we therefore see that $g$ is a decreasing function of $x$ in the immediate vicinity of $x^*$. Thus, since $x_2 > x^*$, we must have $g(x_2) < g(x^*) = 0$, and we can conclude that the second term in the RHS must be positive. Lastly, since $x^*$ is evolutionarily unstable, we must have $\frac{\partial^2 f}{\partial x_1^2}(x_1 ; x^*)\bigg{|}_{x_1 = x^*} > 0$ by the stability criterion. Thus, assuming $f$ is smooth, for $x_2$ sufficiently close to $x^*$, it must be the case that $\frac{\partial^2 f}{\partial x_1^2}(x_1 ; x_2)\bigg{|}_{x_1 = x_2} > 0$. Thus, the third term of the RHS is also positive, meaning that $f(x_1 ; x_2) > 0$, and that $x_1$ can thus invade $x_2$. An exactly analogous argument shows that $f(x_2 ; x_1) > 0$, thus completing the proof.
\end{proof}
Mutual invasibility of $x_1$ and $x_2$ results in a so-called `protected polymorphism' in which the population harbours some members with trait value $x_1$ and some members with trait value $x_2$. This can be shown in some cases to lead to further divergence where the polymorphisms grow further apart in trait space while both being maintained in the population. Thus, the population appears to have `branched' from a monomorphic state to a dimorphic state in trait space. Due to this reason, points of type \circled{B} are called \emph{branching points}, and the population is said to have undergone \emph{evolutionary branching} once it has gone from a monomorphism to a dimorphism. Note that once a population has branched, equation \eqref{AD_canonical_eqn} no longer describes the population, since it is no longer monomorphic - We instead need a system of \emph{two} equations, one for each morph.\\
In adaptive dynamics, the name of the game is thus formulating reasonable guesses for $B(x)$ and $f(y;x)$. For example, one common choice for modelling asexual resource competition is $B(x) \equiv 1$ and the invasion fitness function:
\begin{equation}
	\label{AD_cts_logistic_invasion_fitness}
	f(y:x) = 1 - \frac{\alpha(x,y)K(x)}{K(y)}
\end{equation}
where $\alpha(x,y)$ is called the \emph{competition kernel} and $K(x)$ is called the \emph{carrying capacity function} (formulated in analogy with Lotka-Volterra competition or the logistic equation). Invasion fitnesses are sometimes derived from more mechanistic processes (such as explicitly formulating birth/death functions), but the conceptual idea is the same regardless of how complicated these functions may be, and evolutionary branching easily arises in a very wide range of ecological scenarios as long as the frequency-dependence of the selective force is strong enough \citep{doebeli_evolutionary_2000,doebeli_adaptive_2011}.

\section{The PDE approach}

A slightly more general approach to adaptive diversification is to relax the assumption of separation of timescales while keeping the assumption of infinite population size. In this case, we instead wish to formulate an equation for the distribution $\phi(u)$ of trait values in $\mathcal{T}$. We thus seek PDEs of the form:
\begin{equation*}
	\frac{\partial \phi(u)}{\partial t} = F(u,\phi(u),t)
\end{equation*}
In PDE models, diversification shows up as patterns (in the Turing sense, as covered in Chapter \ref{chap_Turing}), and the existence of a polymorphism corresponds to a multimodal solution for $\phi(x)$. The functional form of $F$ is often formulated through biological principles. For example, in analogy to the logistic equation, one could postulate the continuous version:
\begin{equation}
	\label{cts_logistic}
	\frac{\partial \phi(u)}{\partial t} = r\phi(u)\left(1 - \frac{\phi(u)}{K(u)}\int\limits_{\mathcal{T}}\alpha(u,v)\phi(v)dv\right)
\end{equation}
where $r$ is a growth rate, $K(u)$ gives the carrying capacity of the environment for individuals with phenotype $u$, and $\alpha(u,v)$ is a competition kernel which determines the effect of an individual with phenotype $v$ on the growth of an individual with phenotype $u$, and the integral thus gives a measure of the effective density experienced by individuals with phenotype $u$. $\alpha(u,v)$ is generally chosen such that strength of competition decreases with phenotypic distance (for example, $\alpha(u,v) = \exp(|u-v|)$), in analogy with niche partitioning. Note that this is an integrodifferential equation, and as such, is usually not easy to solve for most functional forms of $\alpha(u,v)$. We thus need to resort to numerical methods to solve such equations. \\
PDE models can also incorporate space more easily than the classic adaptive dynamics approach. For example, Doebeli \textit{et al.} propose a spatial model of resource competition in which we track the density $\phi(x,u)$ of $u$-phenotype individuals at the location $x$ given by the PDE:
\begin{equation}
	\begin{split}
		\label{spatial_PDE}
		\frac{\partial \phi(x,u)}{\partial t} = r\left(\int\limits_{\mathcal{T}}B(v)\phi(x,v)dv - \frac{\phi(x,u)}{K(x,u)}\int\limits_{\mathcal{S}}\int\limits_{\mathcal{T}}\alpha_p(u,v)\alpha_s(x,y)\phi(y,v)dvdy\right) \\ +m\left(\int\limits_{\mathcal{S}}D(x,y)\phi(y,u)dy - \phi(x,u)\right)
	\end{split}
\end{equation}
where $B(v)$ is a birth kernel that specifies births and mutations, $K(x,u)$ is a carrying capacity function that varies with both phenotype and geographic location, $\alpha_p(u,v)$ describes how strength of competition varies with phenotypic distance, $\alpha_s(x,y)$ describes how strength of competition varies with spatial (geographic) distance, $m$ is a migration rate, and $D(x,y)$ is a dispersion kernel that describes the probability that an individual at location $y$ will migrate into location $x$.

PDE models sacrifice interpretability for increased generality. While these models can include space as well as incorporate polymorphic populations more easily, the equations themselves generally have to be solved numerically, and it is more difficult to gain understanding as to why a certain behavior is observed. Nevertheless, mathematically sophisticated techniques have shown that multimodality remains a robust phenomenon for a very wide class of PDE models \citep{elmhirst_pod_2008,doebeli_adaptive_2011}.

\section{Stochastic individual-based models}

The most general approach to modelling adaptive diversification is through individual-based models. In such models, birth rates, death rates, and interaction rules are specified for each individual, and system-level properties are observed by letting the system evolve. Tools such as the Gillespie algorithm are useful here in order to effectively implement simulations. Here, one can include all sorts of complications if they want, and view the outcome. This affords flexibility that PDE models and the classic invasion fitness approach do not, but also leads to even less interpretability. In theory, IbMs can be analyzed as stochastic processes using the ideas covered in chapter \ref{chap_1D_processes}, though depending on the model, the mathematics could get very complicated. As such, depending on complexity, in many cases, IbMs may be best viewed (in my opinion) as in silico `experiments'.

\section{An example of a prediction of adaptive diversification}

Consider the continuous version of the logistic equation given by the invasion fitness \eqref{AD_cts_logistic_invasion_fitness} in adaptive dynamics and the PDE \eqref{cts_logistic} for PDE models. Assume that we have $\alpha(u,v) = \exp{-(u-v)^2/(2\sigma^{2}_{\alpha})}$ and $K(u) = K_{0}\exp{-(u)^2/(2\sigma^{2}_{K})}$, \emph{i.e.} Gaussian carrying capacity and Gaussian competition kernel. Then, it is known that diversification fails (\emph{i.e.} the population remains monomorphic) if and only if:
\begin{equation}\label{AD_monomorphic_condn}
	\sigma_K<\sigma_{\alpha}
\end{equation}.
This result arises from basic calculus using adaptive dynamics, but what does it mean? I argue here that there are two distinct biological interpretations to be made, which I present below:

\subsection{Habitat Filtering}
One way that \ref{AD_monomorphic_condn} can be satisfied is if $\sigma_K$ is very small. Biologically, this means that the environment is such that only some very particular morphs are viable, and the limit where $\sigma_K = 0$ corresponds to a case where only a single morph is environmentally viable. Ecologists are well-acquainted with this notion, and refer to it as `habitat filtering', the phenomenon in which the habitat itself `selects' for certain phenotypes due to particular limiting abiotic factors.

\subsection{Competition and character displacement}
A second way to satisfy \ref{AD_monomorphic_condn} is if $\sigma_{\alpha}$ is very large. Biologically, this means that competition occurs across a wide range of phenotypes, and the limit where $\sigma_{\alpha} = \infty$ corresponds to frequency-independent competition. In other words, diversification can fail if competition cannot be alleviated through character displacement, \emph{i.e} selection is not disruptive (or has a very weak disruptive component). This could happen if, for example, the morphs are competing for a resource that has no alternatives and can only be acquired in a single way (Ex: Competition for sunlight in plants).


\section{Diversity in multi-dimensional trait space}\label{high_dim}
Organisms rarely (if ever) vary along a single independent phenotypic axis. However, adaptive dynamics in multi-dimensional trait space is not very easy to deal with due to mathematical technicalities that arise with identifying neccessary and sufficient criteria for evolutionary singularities to be branching points \citep{doebeli_adaptive_2011,leimar_multidimensional_2009}. As such, adaptive diversification in multi-dimensional trait spaces is usually studied through PDE models or individual-based models \citep{ispolatov_individual-based_2016}.  The motivation for these studies arises from the classic notions of competitive exclusion and limiting similarity, with position in trait space being thought of as analogous to the (Hutchinsonian) ecological niche.\\
Broadly, the results of these studies align with intuitive expectations: As the dimensionality of the trait space increases, expected diversity increases. For one-dimensional adaptive diversification, strong frequency-dependence is needed to ensure coexistence of multiple morphs/branches at equilibrium. Using the continuous logistic equation \eqref{cts_logistic} with Gaussian forms of $K(x)$ and $\alpha(x,y)$, Doebeli and Ispolatov have shown that as the dimensionality of the trait space increases, progressively weaker frequency dependence is sufficient to maintain multiple morphs at equilibrium \citep{doebeli_complexity_2010}. Furthermore, they have also shown that for a fixed level of frequency-dependence, the probability of having multiple morphs at equilibrium increases as the dimensionality of the trait space increases. Other authors have since arrived at broadly similar conclusions in more general settings \citep{debarre_multidimensional_2014,svardal_organismal_2014}. While these studies say that coexistence is \textit{easier} in higher dimensions, they do not comment on how many morphs are expected to coexist at equilibrium. This has only been addressed by a recent study \citep{doebeli_diversity_2017} which used adaptive dynamics for a continuous analogue of the Lotka-Volterra competition equations to attack the question. They find that all else being equal, the number of coexisting morphs should increase exponentially with the dimensionality of the trait space that the organisms compete in. These authors also repeat this analysis using PDE models (Equation \eqref{cts_logistic} in particular) and computational individual-based simulations in place of adaptive dynamics and claim to find the same broad results.
