%Packages

%%%%%%%%%%%%%
%Bibliography stuff
\usepackage[utf8]{inputenc}
\usepackage[english]{babel}
\usepackage{csquotes}
\usepackage[
sorting=nyc,
backend=biber, %backend to use
style=authoryear, %citation style
uniquelist=false,
natbib,
block=ragged,
maxcitenames=2,
maxbibnames=99]{biblatex}


%manually enumerate the bibliography even if using author-year
% \defbibenvironment{bibliography}
%   {\begin{enumerate}}
	%   {\end{enumerate}}
%   {\item}

% Use last name, initial. unless this is non-unique. If this is non-unique, use full first name.
%From https://tex.stackexchange.com/a/631580
\DeclareNameFormat{always-init}{%
	\ifnumequal{\value{uniquename}}{2}
	{\usebibmacro{name:family-given}
		{\namepartfamily}
		{\namepartgiven}
		{\namepartprefix}
		{\namepartsuffix}}
	{\usebibmacro{name:family-given}
		{\namepartfamily}
		{\namepartgiveni}
		{\namepartprefix}
		{\namepartsuffix}}%
	\usebibmacro{name:andothers}}



\DeclareNameAlias{author}{always-init}
\DeclareNameAlias{editor}{always-init}

\ExecuteBibliographyOptions{useprefix=true}
\DeclareSortingNamekeyTemplate{
	\keypart{\namepart{family}}
	\keypart{\namepart{prefix}}
	\keypart{\namepart{given}}
	\keypart{\namepart{suffix}}}


%Sort by name-year-cite order (biblatex default is to sort by name-year-title).
\DeclareSortingTemplate{nyc}{
	\sort{
		\field{presort}
	}
	\sort[final]{
		\field{sortkey}
	}
	\sort{
		\field{sortname}
		\field{author}
		\field{editor}
		\field{translator}
		\field{sorttitle}
		\field{title}
	}
	\sort{
		\field{sortyear}
		\field{year}
	}
	\sort{\citeorder}
}

%Clear some things I do not want going into the bibliography
%from https://tex.stackexchange.com/a/89848
\AtEveryBibitem{\clearfield{month}}
\AtEveryBibitem{\clearfield{day}}
\AtEveryBibitem{\clearfield{url}}
\AtEveryBibitem{\clearfield{urlyear}}
%\AtEveryBibitem{\clearfield{note}}


%%%%%%%%%%%%%%%%
%Add a new command for citing things with possesive apostrophe. For example, if I want to write:  Smith's (2023) big idea, the apostrophe needs to go after the author but before the year
%From https://tex.stackexchange.com/a/537765
%The relevant command is \posscite
\DeclareNameWrapperFormat{labelname:poss}{#1's}

\newrobustcmd*{\posscitealias}{%
	\AtNextCite{%
		\DeclareNameWrapperAlias{labelname}{labelname:poss}}}

\newrobustcmd*{\posscite}{%
	\posscitealias
	\textcite}

\newrobustcmd*{\Posscite}{\bibsentence\posscite}

\newrobustcmd*{\posscites}{%
	\posscitealias
	\textcites}
%%%%%%%%%%%%%%%%%%%%%%%%%%%%%%%%%%%%%%%%%%%%%%%%%%

\usepackage{ragged2e} %for justifying text

\usepackage{amsthm, amsmath, amssymb} % Mathematical typesetting
\usepackage{mathrsfs} %fancy fonts for sigma-algebras
\usepackage{physics} %some physics symbols
\usepackage{float} % Improved interface for floating objects
\usepackage{newfloat} % Declare your own custom floats
\usepackage{graphicx, multicol} % Enhanced support for graphics
\usepackage{xcolor} % Driver-independent color extensions
\usepackage{framed}
\usepackage[normalem]{ulem} %underlining
\usepackage{listings}

\usepackage[toc, title]{appendix} %Appendix


\usepackage[labelfont=bf]{caption} %bold captions on figures and tables
\usepackage{titlesec} %title formatting


%%%%%%%%%%%%%
%Math stuff
\usepackage{amsfonts}
\usepackage{enumitem}
\usepackage{mathtools}
\usepackage{multicol}
\usepackage{color,soul}


\usepackage{fancyhdr} %fancy headers
\usepackage{afterpage}

%%%%%%%%%%%%%
%for tables
\usepackage{makecell,tabularx}
%\setlength{\extrarowheight}{12pt} %additive padding
\renewcommand{\arraystretch}{2} %multiplicative padding 
\renewcommand\theadfont{\small\bfseries}
\usepackage{rotating}
\usepackage{setspace}

%For pseudocode
\usepackage[boxed]{algorithm2e}
\DontPrintSemicolon

%For dummy text
\usepackage{lipsum}


%A bunch of definitions that make my life easier
\newtheorem{theorem}{Theorem}[section]
\newtheorem{corollary}{Corollary}[section]
\theoremstyle{definition}
\newtheorem*{definition}{Definition}
\newtheorem{example}{Example}
\newtheorem*{note}{Note}
\newtheorem*{claim}{Claim}
\newtheorem*{lemma}{Lemma}
\newcommand{\bproof}{\bigskip {\bf Proof. }}
\newcommand{\eproof}{\hfill\qedsymbol}
\newcommand{\Disp}{\displaystyle}
\newcommand{\qe}{\hfill\(\bigtriangledown\)}
\setlength{\columnseprule}{1 pt}

\newcommand{\bignorm}[1]{\left\lVert#1\right\rVert} %for big norm symbol

%for characters inside circles
%syntax is \circled{character}
\usepackage{tikz}
\newcommand*\circled[1]{\tikz[baseline=(char.base)]{
		\node[shape=circle,draw,inner sep=1pt] (char) {#1};}}

% Defines the `mycase` environment for cases in proofs
\newcounter{cases}
\newcounter{subcases}[cases]
\newenvironment{mycase}
{
	\setcounter{cases}{0}
	\setcounter{subcases}{0}
	\newcommand{\case}
	{
		\stepcounter{cases}\textbf{Case \thecases.}
	}
	\newcommand{\subcase}
	{
		\par\indent\stepcounter{subcases}\textit{Subcase (\thesubcases):}
	}
}
{
	\par
}
\renewcommand*\thecases{\arabic{cases}}
\renewcommand*\thesubcases{\roman{subcases}}
%%%%%%%%%%%%%%%%%%%%%%%%%%%%%%%%%%%%
%For easily making figures
%syntax is:
%\myfig{scaling_factor}{name_of_file}{caption}{label}
\newcommand{\myfig}[4]{\begin{figure}[H]\centering \begin{center} \includegraphics[width=#1\textwidth]{#2} \caption{#3} \label{#4} \end{center} \end{figure}}

% horizontal line across the page
\newcommand{\horz}{
	\vspace{-.4in}
	\begin{center}
		\begin{tabular}{p{\textwidth}}\\
			\hline
		\end{tabular}
	\end{center}
}

%%%%%%%%%%%%%%%%%%%%%%%%%%%%%%%%%5
%for making a box
%modified from https://stackoverflow.com/a/77034123
\usepackage[most]{tcolorbox}

\newtcolorbox[auto counter,]{fancybox}[3][]{
	arc=5mm,
	lower separated=false,
	fonttitle=\bfseries,
	colbacktitle=black!10,
	coltitle=black,
	colupper=black,
	enhanced,
	attach boxed title to top left={xshift=1.8cm,
		yshift=-3mm},
	colframe=black,
	colback=black!10,
	title=#2 \thetcbcounter : #3,#1,breakable}

%%%%%%%%%%%%%%%%%%%%%%%%%%%%%%%%%%%%%%%%%%%%%%%%
%Resize the summation symbol
%syntax is \sum[size] 
\newlength{\depthofsumsign}
\setlength{\depthofsumsign}{\depthof{$\sum$}}
\newlength{\totalheightofsumsign}
\newlength{\heightanddepthofargument}
\newcommand{\bigsum}[1][1.4]{% only for \displaystyle
	\mathop{%
		\raisebox
		{-#1\depthofsumsign+1\depthofsumsign}
		{\scalebox
			{#1}
			{$\displaystyle\sum$}%
		}
	}
}

%Add a custom strut to increase vertical space given to equations when combining with underbrace
%from https://tex.stackexchange.com/a/13864
\newcommand*\mystrut[1]{\vrule width0pt height0pt depth#1\relax}


%for cross-referencing between files
%from https://tex.stackexchange.com/a/14365
\usepackage{xr-hyper}

%overleaf needs some extra hacks to work with the xr and xr-hyper packages.
%from https://www.overleaf.com/learn/how-to/Cross_referencing_with_the_xr_package_in_Overleaf for overleaf specific
%instructions
\makeatletter
\newcommand*{\addFileDependency}[1]{% argument=file name and extension
	\typeout{(#1)}% latexmk will find this if $recorder=0
	% however, in that case, it will ignore #1 if it is a .aux or 
	% .pdf file etc and it exists! If it doesn't exist, it will appear 
	% in the list of dependents regardless)
	%
	% Write the following if you want it to appear in \listfiles 
	% --- although not really necessary and latexmk doesn't use this
	%
	\@addtofilelist{#1}
	%
	% latexmk will find this message if #1 doesn't exist (yet)
	\IfFileExists{#1}{}{\typeout{No file #1.}}
}\makeatother

\newcommand*{\myexternaldocument}[1]{%
	\externaldocument{#1}%
	\addFileDependency{#1.tex}%
	\addFileDependency{#1.aux}%
}

%hyperref should be imported AFTER xr-hyper
\usepackage[final, colorlinks = true, 
linkcolor = black, 
citecolor = black,
filecolor = black,
hypertexnames = false,
breaklinks=true]{hyperref} % For hyperlinks in the PDF